% Hamad Medical Corporation
% Georges Younes

\chapter{Potential Beneficiaries}\label{chp:beneficiaries}
We plan to disseminate our results to the research community, the medical clinical community, and to the public at large. Dissemination to the research community will be primarily through technical papers submitted to scientific journals and presented at international conferences. We envision publication in journals related to finite element technology, physical modeling, geometry processing, and haptics technology; journals related to translational research (how innovations are translated into practical applications), virtual reality, and integrative medical robotics technologies; as well as clinical journals related to robotic surgery and to the validation/assessment of training simulators in urological and related procedures. We have also dedicated portions of the funds in the project to participate in international conferences where we present the work directly. Our papers, presentations, and video demonstrations will also be available on a publicly available web site for the project.% and accessible through the \acr{QSTP} portal.

Dissemination to the local, regional, and international clinical medical community will be performed through a number of mechanisms. The team of surgeons available through \acr{hmc}'s network, particularly those who will be directly involved in the planning and validation of the simulator, will be disseminating the results to a large group of local surgeons, ensuring adequate local coverage of our research activities. In addition, the \acr{pi}'s of the project will organize annual workshops in Qatar to share the results of the proposed work with the community and various government agencies and industries interested in robotic surgery training. Qatar, being the first to bring robotic surgery technology and training to the middle east, is likely to be the hub of activities related to the demonstration, evaluation, and assessment of this technology in this region, reaching almost all middle-east surgeons interested in \acr{mis}. A simulator that pushes the state of the art will also place Qatar on the international map in developing surgical training technology. To this end, we will also demonstrate our system through Mimic Technologies, Inc. Mimic regularly participates and has demonstration booths in international meetings and symposia related to \acr{mis} and attended by urologists interested in robotic-assisted procedures. A demonstration of our radical prostatectomy simulator there would be a natural vehicle for dissemination, since only software has to be sent to these demonstrations. The senior members of the project also plan to organize a mini-symposium associated with one of these meetings (\acr{mmvr} or \acr{aua}) during the final year of the project.

Dissemination to the public will be through actual demonstration of the system and a web site showcasing our results. The small footprint and relative portability of the system make it possible to load it on a truck for road shows, and we plan on doing so during the last year of the project. The safety of the virtual environment will allow high school students to experience the system first-hand. This experience will hopefully excite them about the potential of simulation technology and the promise of robotic \acr{mis} surgery, and inspire them to consider careers in science, technology, and surgery. A web site will also be launched and will feature videos and images of the system and allow communication with the broad community.
