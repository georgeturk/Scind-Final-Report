% Hamad Medical Corporation
% Georges Younes

\chapter{Recommendations}\label{chp:recommendations}

\begin{enumerate}
  \item extensive use of simulators by residents
\end{enumerate}

Three follow-on continuation activities are envisioned. The first one is to commercialize the system. There is significant demand for improved training simulators now that simulation in the \acr{usa} as well as in other parts of the world. Hospitals are rushing to set simulation centers and we expect this demand to increase worldwide. The demand will be specifically for more physically-realistic, validated systems of the kind we are developing. While the exact commercialization mechanism will have to be worked out in due time, there are a number of possible avenues. One is to have the graduate students and personnel involved in the project form a start-up company building directly on the system developed, with revenue-sharing agreements with the institutions that will hold the \acr{ip}: \acr{hmc}, \acr{aub}, and \acr{unc}. Another is to license the system to one of the existing companies that are currently developing and selling commercial simulators. \acr{isi} may also be directly interested in licensing and packaging the simulator with their console.

The other continuation effort will target procedures besides prostatectomy. Cardiac, thoratic, fundoplication, gynecologic, and other urologic procedures are now being performed robotically creating the need for training targeted to them. Similarly, other robotic surgical workstations (besides \acr{isi}'s da Vinci) may become commercially available in the early part of the next decade (smaller, mobile, more dexterous, \etc) and retargeting our simulation technology to support them is an obvious activity. From a technology perspective, the demand will continue for more detailed and higher fidelity tissue mechanics models, higher resolution geometric details, and interaction with more sophisticated instruments, continuing to push the limits of what is feasible to simulate in real-time.

Finally, the next stage in advancing the state surgical simulation is to use the technology intra-operatively. In this scenario, the models used in the simulation are patient-specific models generated pre-operatively from \acr{ct} and \acr{mri} scans. The tissue properties of the model are also updated during the procedure from information obtained from intra-operative \acr{ct} scans or ultrasound scans. Instead of using nominal tissue mechanics model parameters, patient-specific material properties will be computed and updated to reflect the observed specific behavior of the patient’s tissues. In this environment, the surgeon will now have access to stereo endoscope data, \acr{ct}/\acr{us} scans, as well as to the predictions from the simulator regarding stresses, strains, tissue damage, location of surgical margins, \etc to make decisions. This \enquote{surgical cockpit} is information rich and a number of human-computer interaction issues will have to be worked out in order to present this information effectively, and reduce the sensory overload on the surgeon. This exciting next phase of the research will demand significantly more computational capabilities to be available to the operating room.
