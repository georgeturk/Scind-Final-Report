% Hamad Medical Corporation
% Georges Younes

\chapter{Literature Review}\label{chp:literature}

Simulation of surgical cutting is a problem of tremendous interest as it is a key building block in many simulation applications. Various aspects of surgical procedures, including construction of geometric representations, texture generation, topological modifications, geometric intersection and query operations, collision detection and processing, physical models for deformations, integration with input devices for separating and pushing deformable tissue models, output rendering using \acr{3d} stereoscopic and haptic displays, and many other related aspects, have been and continue to be addressed by various researchers. Realtime simulation represents an extremely challenging class of computational science simulations where both high performance and absolute robustness are essential requisites for effectiveness \autocite{younes:cbm:2013}. To be used as tools for either procedure rehearsal or training, surgical simulators must provide support for realistic and consistent geometric and mechanical behavior \autocite{farhat:hsmr:2012} suitable for real-time interactions. Even a single contact that is not correctly detected and resolved, anywhere in space and at any time step, can result in completely erroneous and visually distracting simulation.

The challenges of these simulations come from the need for:
\begin{enumerate*}[(1)]
  \item interaction with non-linear deformable materials that undergo topological changes due to incisions and resections;
  \item geometrically rich tool-
tissue contact and interaction; and
  \item dynamically changing tissue-tissue contact interactions.
\end{enumerate*}
To ensure robustness in this environment, geometric computations and algorithms must interact at a fine-grained level with a finite element model of non-linear elastic deformations to resolve the dynamic contact conditions and update model deformations accordingly. Vertex-face and edge-edge contacts are identified and used to constrain the space of allowable deformations. These computations must be performed in real-time at several frames per second on models involving several tens of thousands of elements, pushing the limits of robust simulation in a high-performance computational simulation.

A couple of paragraphs on commercial systems.

\hrule

Cotin \etal~\autocite{cotin:tvc:2000} proposes physical elastic models based on linear elasticity and finite element discretizations. They introduce a tensor-mass formulation which elegantly generalizes the spring-mass models to simplify the implementation of the numerical model. However, the system does not handle large displacements such as those occurring during surgical manipulation of tissue. Nonlinear kinematics are needed to properly model such behavior.

Ganovelli \etal~\autocite{ganovelli:tvc:2001} tackles the problem of dynamically modifying the topology of a \acr{3d} tetrahedral mesh as it gets cut by a surgical tool moving along a planar surface. Ad-hoc tetrahedralization of various cut configurations, stored in a look-up table addressed by the intersected edges, are introduced in the mesh. A multi-resolution mesh allows region interference queries to be performed efficiently. However the algorithm only handles cut tetrahedra after they have been completely split and cannot handle the continuous cuts that occur in surgical cutting.

Bielser \etal~\autocite{bielser:gm:2004} introduces a state machine representation for handling continuous cutting of tetrahedral meshes. The algorithm allows tetrahedra to be continuously subdivided and generates a compatible tetrahedralization. A limitation of this method however is in the way it handles the cases where a cut intersects an edge in multiple locations. A recursive continuation scheme is proposed for subdividing the parent tetrahedra that have double-cut edges and applying the state machine rules to the children tetrahedra. Unfortunately, guaranteeing the correct continuation of the cut surfaces using this recursive strategy requires complex processing and results in an explosion in the resulting number of subdivided tetrahedra that could limit the ability to perform real time physical simulation of the cut mesh.

Teschner \etal~\autocite{teschner:cgf:2005} surveys various methods for collision detection in deformable meshes including bounding volume hierarchies, distance fields, spatial partitioning, image-space techniques, and stochastic methods. They illustrate the use of hardware rendering methods and stochastic methods for collision detection in liver and intestinal surgery simulations in some constrained settings. However, they conclude that there exist no general or optimal approach for collision detection and the quality of the generated approximation, the generality of the detection, and the target hardware (\acr{cpu}/\acr{gpu}) need to be analyzed to choose the most adequate procedures for a given setting.

Wu \etal~\autocite{wu:tvc:2005} uses linear finite element discretization of the \acr{3d} volume with a condensation method that eliminates the degrees of freedom that are not directly manipulated during the surgical procedure. However the methods are limited to linear displacement and deformations and would not scale to the case of large nonlinear displacements typical in general surgical scenarios. Nakao \etal~\autocite{nakao:jms:2006} also use linear finite elements coupled with a particle-based models for preoperative planning but the system is also limited to small displacements and rotations.

Pietroni \etal~\autocite{pietroni:tvc:2009} embed a \acr{3d} tetrahedral mesh into a regular hexahedral cubical grid and performs splitting operations on the embedding grid where they are easier to perform. The hexahedral grid provides a tessellation for the evolving boundary surface surface embedded in a deforming space. The primary difficulty with this algorithm however is the need to adapt the granularity of the grid as not to prevent general topological cuts from registering on it. For example initiating a cut by intersecting faces or edges of the model inside a cubical grid cell may require adaptive grid refinement adding substantial complexity and computational time.

Dick \etal~\autocite{dick:tvcg:2011} couples the geometric cutting procedure with the numerical solution scheme of the physical model to allow tighter interaction between the two procedures. Both geometric intersection queries and solution use hexahedral multigrid scheme that can operate at multiple levels of granularity. The induced cut discontinuities are embedded in the geometric multigrid hierarchy, and propagated to the fast algorithms for updating the resulting systems of equations. However the system is not intended for real time interactive operations as it requires several seconds of computation to resolve every time step in the simulation.

Zerbato \etal~\autocite{zerbato:cars:2011} use \acr{gpu} hardware to solve the physics equations governing deformable models. The computations are mapped to vertex and fragment shaders of the rendering pipeline in order to obtain real time performance. However the work is limited by a number of simplifications introduce in the computations. The model is not based on an underlying \acr{3d} volumetric representation and one cannot therefore push a tool tip at an arbitrary point on a boundary face. In the case of cutting, the cut surfaces cannot be properly resolved or interacted with during the simulations.

Seiler \etal~\autocite{seiler:tvc:2011} describe an adaptive octree based approach for interactive cutting of deformable objects.
The algorithm is simplified by assuming that pieces of material are completely excised from a volumetric object in an atomic procedure. The cutting blade geometry is transformed back into material space and intersected with a surface geometry to allow strongly deformed geometries, even involving inverted and degenerate elements, to be cut. However the fundamental restriction to non-progressive cuts limits the use of this method to general surgical cuts that involve a smooth sequence of cutting movements of incision tools.

Wu \etal~\autocite{wu:tvc:2013, wu:cgf:2015} approximate a \acr{3d} volumetric tetrahedral model by coarse composite semi-regular hexahedral finite elements that group blocks of elements together, thereby reducing the numbers of simulation degrees of freedom and thus trading performance for a decrease in accuracy. Collision detection computations can be also be made faster by using the smaller number of coarser regular elements. Jia \etal~\autocite{jia:cst:2017} also use voxels connected by links and embedded inside adaptive octree meshes to model deformable objects. However it is not always clear how effective the quality of these coarser approximations are for complex geometries or when the physics of collision response are needed beyond penetration depth.

Paulus \etal~\autocite{paulus:tvc:2015} propose a method that introduces new vertices inside all tetrahedra that are affected by a cut, either fully partially cut. A sequence of topological flip operations and triangulation of the generated facets dual to cut edges produces a mesh that contains a triangulation of the surface trajectory of the cutting tool. The mesh is then separated along the faces and vertices of this triangulation. Some special care is needed to handle cuts that intersect object boundaries. The quality of the resulting tetrahedra can be controlled by setting appropriate thresholds for snapping vertices to the cut surface to avoid small edges, and for repositioning inserted vertices to improve element shapes. However, and even though this technique works well for simple configurations, it is not clear how to resolve cut paths with high curvature that may intersect  model edges in more than one location.

Weng and Sourin~\autocite{weng:sym:2018} describe a dual mesh system which uses a high-resolution visual mesh and a low-resolution collision mesh and introduce a visual-collision binding between them, and are able to extend the spatially reduced framework to support cutting. The cutting paths are detected on the collision triangles and then mapped to local 2D coordinates systems in which the intersections between visual mesh and the cutting paths are calculated.  The visual-collision binding is updated locally after cutting. However, the method is inherently limited to two-manifold triangular meshes and it is not clear how the generalization to 3D volumetric tetrahedral models can be performed.

Peng \etal~\autocite{peng:mta:2019} use a combination of a particle-based representation in the interior and mesh-based representation near the boundary where user interaction takes place to simulate cornea cutting and deformation. However the system is limited in the kind of tool-tissue interaction it can handle including sliding or pulling away from the model, and interaction of models with non-point tools. The scalability of the methods may also limit their use with more complex and refined models.

None of these prior methods provided the benefits of accurate finite-element methods integrated with continuous collision detection to perform robust simulation.
