% Georges Younes
% Hamad Medical Corporation

\documentclass[english,a4paper,oneside,leqno,11pt]{memoir}

\usepackage[utf8]{inputenc}
%\usepackage[T1]{fontenc}
%\usepackage{excludeonly}
%\DisemulatePackage{setspace}
%\usepackage[nodisplayskipstretch]{setspace}
%\usepackage{marginnote}
\usepackage[counterclockwise]{rotating}
\usepackage{graphicx}
\usepackage[export]{adjustbox}
\graphicspath{{figures/}}
\usepackage[english]{babel}
\usepackage[pangram]{blindtext}
%\usepackage{csquotes}
%\usepackage{lipsum}
\usepackage{caption}
\usepackage[export]{adjustbox}
%\usepackage{type1cm}
%\usepackage{ragged2e}
%\usepackage{microtype}
\usepackage{listings}
\usepackage{fix-cm}
\usepackage{amsmath,amssymb,amsxtra,amsbsy,amsfonts}
\usepackage{bm,mathtools}
\usepackage[ruled]{algorithm}
%\usepackage{algorithmicx}
\usepackage[noend]{algpseudocode}
\usepackage[fixpdftex,usenames,dvipsnames,table]{xcolor}
%\usepackage{indentfirst}
%\usepackage{paralist}
\usepackage{multirow}
\usepackage{multicol}
\usepackage{xspace}
\usepackage{tikz,tikz-3dplot,pgfplots}
\usepackage{ifthen}
%\usepackage[nocompress]{cite}
\usepackage{varioref}
\usepackage{booktabs}
\usepackage{tabularx}
\usepackage[binary-units]{siunitx}
\usepackage{authblk}
\usepackage{pageslts}
\usepackage[autostyle]{csquotes}
\usepackage{pdfpages}
\usepackage{longtable}
\usepackage{enumerate}
\usepackage[inline,shortlabels]{enumitem}
\usepackage{soul}
\usepackage[breaklinks]{hyperref}
%\usepackage[pagebackref,breaklinks]{hyperref}
\usepackage{memhfixc}
\usepackage[all]{hypcap}
\usepackage[toc,acronym,symbols,nonumberlist]{glossaries}
%\usepackage[toc,acronym,symbols,nonumberlist]{glossaries-extra}
\usepackage[defernumbers=true,backend=biber,style=numeric-comp,sorting=none,isbn=false,url=false,doi=false,maxbibnames=99]{biblatex}
\addbibresource{report.bib}
\addbibresource{proposal.bib}
\addbibresource{project.bib}
%\bibliography{rep

%% helper commands for cover page
\newcommand{\xtitle}[1]{\gdef\xtitle{#1}\title{#1}}
\newcommand{\xnprp}[1]{\gdef\xnprp{#1}}
\newcommand{\xdate}[1]{\gdef\xdate{#1}\date{#1}}
\newcommand{\xstartdate}[1]{\gdef\xstartdate{#1}}
\newcommand{\xenddate}[1]{\gdef\xenddate{#1}}
\newcommand{\xsummary}[1]{\gdef\xsummary{#1}}

%%% https://tex.stackexchange.com/questions/74707/how-to-concatenate-strings-into-a-single-command
\usepackage{etoolbox}

\gdef\xinvestigators{}
\gdef\xaffiliations{}

\makeatletter%
%\newcommand\uptag[1]{\textsuperscript{#1}}
%\newcommand\uptag[1]{\textsuperscript{\@fnsymbol{#1}}}
% https://tex.stackexchange.com/questions/63895/click-to-go-to-an-anchored-line
\newcommand\uptagi[1]{\hyperlink{#1}{\textsuperscript{\textit{\@alph{#1}}}}}
\newcommand\uptaga[1]{\hypertarget{#1}{\textsuperscript{\textit{\@alph{#1}}}}}
\makeatother%

\newcommand{\xinvestigator}[2][]{
  \author[#1]{#2}
  \appto\xinvestigators{#2\uptagi{#1}\par}
}

\newcommand{\xaffiliation}[2][]{
  \affil[#1]{#2}
  \appto\xaffiliations{#2\uptaga{#1}\par}
}

%%% toc format
%\setcounter{secnumdepth}{6}
%\setcounter{tocdepth}{6}
\setsecnumdepth{paragraph}
\settocdepth{subsubsection}

%%% page margins
%\setlrmarginsandblock{1.5cm}{7.5cm}{*}
\setlrmarginsandblock{2.5cm}{2.5cm}{*}
%\setlrmarginsandblock{4.5cm}{4.5cm}{*}
\setulmarginsandblock{3.0cm}{4.0cm}{*}
%\setmarginnotes{7.5mm}{6cm}{0cm}
\setmarginnotes{7.5mm}{2cm}{0cm}
\checkandfixthelayout%
%\setlength{\footskip}{1.45\baselineskip}

%%% margin notes format
\renewcommand{\sideparfont}{\raggedright\footnotesize\color[named]{Gray}}
\sideparmargin{outer}

%%% boxed chapter style
%%% https://texblog.org/2012/07/03/fancy-latex-chapter-styles/
%%% https://latex.org/forum/viewtopic.php?t=16624
\makechapterstyle{box}{
  \setlength{\beforechapskip}{0pt}
  \setlength{\afterchapskip}{0.7cm}
  \renewcommand*{\printchaptername}{}
  \renewcommand*{\chapnumfont}{\normalfont\sffamily\huge\bfseries}
  %\renewcommand*{\printchapternum}{}
  \renewcommand*{\printchapternum}{
    \flushleft
    \begin{tikzpicture}
      \draw[fill,color=black] (0,0) rectangle (1cm,1cm);
      \draw[color=white] (0.5cm,0.5cm) node {\chapnumfont\thechapter};
    \end{tikzpicture}
    \vspace{-5ex}
  }
  \renewcommand*{\chaptitlefont}{\normalfont\sffamily\huge\bfseries}
  \renewcommand*{\printchaptertitle}[1]{\flushleft\chaptitlefont##1}
%   \renewcommand*{\printchaptertitle}[1]{
%     \flushleft
%     \begin{tikzpicture}
%       \draw[fill,color=black] (0,0) rectangle (1cm,1cm);
%       \draw[color=white] (0.5cm,0.5cm) node {\chapnumfont\thechapter};
%     \end{tikzpicture}
%     \chaptitlefont##1
%   }
}
\chapterstyle{box}

%%% heading styles
\setsecheadstyle{\normalfont\large\bfseries}
\setsubsecheadstyle{\normalfont\normalsize\bfseries}
\setsubsubsecheadstyle{\normalfont\normalsize\bfseries}
\setparaheadstyle{\normalfont\normalsize\bfseries}
\setsubparaheadstyle{\normalfont\normalsize\bfseries}

%%% table of contents name
\addto\captionsenglish{%
  \renewcommand\contentsname{Table of Contents}%
}

%%% page numbering in header/footer
%%% https://tex.stackexchange.com/questions/227/how-can-i-add-page-of-on-my-document
\makeevenhead{plain}{}{}{}
\makeoddhead{plain}{}{}{}
\makeevenhead{headings}{}{}{}
\makeoddhead{headings}{}{}{}
\makeevenfoot{plain}{\footnotesize NPRP \#\xnprp}{}{\footnotesize\thepage\ (\theCurrentPage/\lastpageref{LastPages})}
\makeoddfoot{plain}{\footnotesize NPRP \#\xnprp}{}{\footnotesize\thepage\ (\theCurrentPage/\lastpageref{LastPages})}
\makeevenfoot{headings}{\footnotesize NPRP \#\xnprp}{}{\footnotesize\thepage\ (\theCurrentPage/\lastpageref{LastPages})}
\makeoddfoot{headings}{\footnotesize NPRP \#\xnprp}{}{\footnotesize\thepage\ (\theCurrentPage/\lastpageref{LastPages})}

%\makefootrule{plain}{\textwidth}{\normalrulethickness}{}
%\makeheadrule{headings}{\textwidth}{\normalrulethickness}{}

%%% toc dots
\renewcommand{\cftchapterdotsep}{\cftdotsep}
\renewcommand{\cftchapterleader}{\cftdotfill{\cftchapterdotsep}}
%\renewcommand{\cftchapterfont}{\normalfont}
%\renewcommand{\cftchapterpagefont}{\normalfont}

%%% compact lists
%%% https://latex.org/forum/viewtopic.php?t=25565
%\usepackage{paralist}
%\let\itemize\compactitem
%\let\enditemize\endcompactitem
%\let\enumerate\compactenum
%\let\endenumerate\endcompactenum
%\let\description\compactdesc
%\let\enddescription\endcompactdesc
%\pltopsep=\medskipamount
%\plitemsep=1pt
%\plparsep=1pt

% Disable indent
\setlength{\parindent}{0pt}
% Paragraph break
\setlength{\parskip}{1ex}

%\usepackage[shortlabels]{enumitem}
% \setlist{
% %  itemindent=,
% %  labelsep=,
% %  labelindent=,
% %  leftmargin=*,
%   listsep=1ex,
%   listparindent=0pt,
%   noitemsep,
%   nosep,
% %  before=
% }
\setlist{leftmargin=*,itemsep=0ex}

%%% float numbering
\numberwithin{figure}{chapter}
\numberwithin{table}{chapter}
\numberwithin{equation}{chapter}
\numberwithin{algorithm}{chapter}
%\numberwithin{lstlisting}{chapter}

%%% float default locations
\setfloatlocations{figure}{htp}
\setfloatlocations{table}{htp}
\setfloatlocations{algorithm}{tbp}
\setfloatlocations{lstlisting}{tbp}

\captionsetup[ruled]{
  strut=off
}
\captionsetup{
  format=plain,
  labelformat=simple,
%  labelsep=colon,
  labelsep=space,
  labelfont=bf,
  justification=raggedright,
  font=small,
  width=\textwidth
}

\newsubfloat{figure}
%\renewcommand{\thesubfigure}{(\arabic{subfigure})}
%\renewcommand{\thesubfigure}{(\emph{\roman{subfigure}})}
\renewcommand{\thesubfigure}{(\emph{\alph{subfigure}})}

%%% various options for hyperref
%http://www.tug.org/applications/hyperref/manual.html#x1-40003
\hypersetup{
  unicode,
  %pdftex,
  %xetex,
  bookmarks,
  bookmarksnumbered,
  bookmarksdepth=subsection,
  colorlinks=true,
  linktoc=all,
  %linkcolor=gray,
  %anchorcolor=gray,
  %citecolor=gray,
  %filecolor=gray,
  %menucolor=gray,
  %runcolor=gray,
  %urlcolor=gray,
  linkcolor=Blue,
  anchorcolor=Blue,
  citecolor=Blue,
  filecolor=Blue,
  menucolor=Blue,
  runcolor=Blue,
  urlcolor=Blue,
  plainpages=false,
  pdfpagelabels,
  hyperfigures,
  hyperindex,
  hyperfootnotes,
  hypertexnames, %page links in the index will stop working (page 86 - lshort)
  pdfborder={0 0 0},
  %pdfstartview={FitH},
  pdfnewwindow,
  %pdfpagetransition={Dissolve},
  %pdfpagelayout=OneColumn,
  %pageanchor=false,
  pdfcreator={},
  pdfproducer={},
  pdftitle={},
  pdfauthor={},
  pdfsubject={},
  pdfkeywords={},
  %http://www.cl.cam.ac.uk/~mgk25/publ-tips/
  %pdfpagescrop={70 28 552 786},
}

%\renewcommand*{\chapterautorefname}{Chapter}
%\renewcommand*{\sectionautorefname}{Section}
%\renewcommand*{\subsectionautorefname}{\sectionautorefname}
%\renewcommand*{\subsubsectionautorefname}{\sectionautorefname}
%\renewcommand*{\paragraphautorefname}{Paragraph}
%\renewcommand*{\subparagraphautorefname}{\paragraphautorefname}
%\newcommand{\subfigureautorefname}{\figureautorefname}

%%% https://tex.stackexchange.com/questions/370991/change-chapter-to-chapter-how-to-do-this-using-autoref
\addto\extrasenglish{%
  \def\chapterautorefname{Chapter}%
  %\def\appendixautorefname{Appendix}%
  \def\sectionautorefname{Section}%
  \def\subsectionautorefname{\sectionautorefname}%
  \def\subsubsectionautorefname{\sectionautorefname}%
  \def\paragraphautorefname{Paragraph}%
  \def\subparagraphautorefname{\paragraphautorefname}%
  \def\subfigureautorefname{\figureautorefname}%
}

%%% sans-serif for chapter/appendix headings in toc
%%% https://tex.stackexchange.com/questions/96899/change-the-chapter-font-in-toc-without-using-tocloft-package
\makeatletter%
\let\originall@chapter\l@chapter
\def\l@chapter#1#2{\originall@chapter{{\sffamily #1}}{#2}}
\let\originall@appendix\l@appendix
\def\l@appendix#1#2{\originall@appendix{{\sffamily #1}}{#2}}
\makeatother%

\nonfrenchspacing%

%%%%%%%%%%%%%%%%%%%%%%%%%%%%%%%%%%%%%%%%%%%%%%%%%%%%%%%%%%%%%%%%%%%%%%%%%%%%%%%%

%%% automatic referencing
%\newcommand\frf[1]{{\bfseries#1}}
\newcommand\frf[1]{#1}
%\labelformat{chapter}{Chapter~\frf#1}
%\labelformat{section}{Section~\frf#1}
%\labelformat{subsection}{Subsection~\frf#1}
%\labelformat{subsubsection}{Subsubsection~\frf#1}
%\labelformat{paragraph}{Paragraph~\frf#1}
%\labelformat{subparagraph}{Subparagraph~\frf#1}
%\labelformat{equation}{(\frf#1)}
%\labelformat{figure}{Figure~\frf#1}
%\labelformat{table}{Table~\frf#1}
%\labelformat{algorithm}{Algorithm~\frf#1}
\labelformat{chapter}{\frf#1}
\labelformat{section}{\frf#1}
\labelformat{subsection}{\frf#1}
\labelformat{subsubsection}{\frf#1}
\labelformat{paragraph}{\frf#1}
\labelformat{subparagraph}{\frf#1}
\labelformat{equation}{(\frf#1)}
\labelformat{figure}{\frf#1}
\labelformat{table}{\frf#1}
\labelformat{algorithm}{\frf#1}

% https://stackoverflow.com/questions/3282319/correct-way-to-define-macros-etc-ie-in-latex
\usepackage{xspace}
\makeatletter
\DeclareRobustCommand\onedot{\futurelet\@let@token\@onedot}
\def\@onedot{\ifx\@let@token.\else.\null\fi\xspace}
\newcommand{\latin}[1]{{\fontshape{ui}\selectfont#1}}

\def\eg{\latin{e.g}\onedot}% \def\Eg{\latin{E.g}\onedot}
\def\ie{\latin{i.e}\onedot}% \def\Ie{\latin{I.e}\onedot}
\def\cf{\latin{c.f}\onedot}% \def\Cf{\latin{C.f}\onedot}
\def\etc{\latin{etc}\onedot}
\def\vs{\latin{vs}\onedot}
\def\wrt{w.r.t\onedot}
\def\dof{d.o.f\onedot}
\def\etal{\latin{et al}\onedot}
\makeatother

%\sisetup{per-mode=symbol,per-symbol=p}
\sisetup{quotient-mode=fraction,range-units=single,range-phrase=--}

%%% color palette
% https://clrs.cc
% https://openglcolor.mpeters.me
\xdefinecolor{Aqua}{rgb}{0.49803922,0.85882353,1.0}
\xdefinecolor{Black}{rgb}{0.06666667,0.06666667,0.06666667}
\xdefinecolor{Blue}{rgb}{0.0,0.45490196,0.85098039}
\xdefinecolor{Cream}{rgb}{0.93333333,0.93333333,0.93333333}
\xdefinecolor{DarkGray}{rgb}{0.33333333,0.33333333,0.33333333}
\xdefinecolor{Fuschia}{rgb}{0.94117647,0.07058824,0.74509804}
\xdefinecolor{Gray}{rgb}{0.66666667,0.66666667,0.66666667}
\xdefinecolor{Green}{rgb}{0.18039216,0.8,0.25098039}
\xdefinecolor{Lime}{rgb}{0.00392157,1.0,0.43921569}
\xdefinecolor{Maroon}{rgb}{0.52156863,0.07843137,0.29411765}
\xdefinecolor{Navy}{rgb}{0.0,0.12156863,0.24705882}
\xdefinecolor{Olive}{rgb}{0.23921569,0.6,0.43921569}
\xdefinecolor{Orange}{rgb}{1.0,0.52156863,0.10588235}
\xdefinecolor{Purple}{rgb}{0.69411765,0.05098039,0.78823529}
\xdefinecolor{Red}{rgb}{1.0,0.25490196,0.21176471}
\xdefinecolor{Silver}{rgb}{0.86666667,0.86666667,0.86666667}
\xdefinecolor{Teal}{rgb}{0.22352941,0.8,0.8}
\xdefinecolor{White}{rgb}{1.0,1.0,1.0}
\xdefinecolor{Yellow}{rgb}{1.0,0.8627451,0.0}

%%% left ruler
%%% https://latex.org/forum/viewtopic.php?t=9072
%\usepackage{eso-pic}
%\usepackage{calc}
%\usepackage{blindtext}
%
%\newlength{\leftrule}
%\newlength{\rightrule}
%\setlength{\leftrule}{2\leftmargin-0.5\marginparsep}
%\setlength{\rightrule}{2\leftmargin+\textwidth}
%\AddToShipoutPicture{%
%   \AtPageLowerLeft{%
%%     \put(\LenToUnit{\leftrule},0){\rule{0.5pt}{\paperheight}}  % Rule on the left
%     \put(\LenToUnit{\rightrule},0){\rule{0.5pt}{\paperheight}}  % Rule on the right
%   }
%}
%\SetBgScale{1}
% \SetBgAngle{0}
% \SetBgColor{gray}
% \SetBgContents{\rule{.4pt}{\paperheight}}
% \SetBgHshift{-9cm}

%%% https://tex.stackexchange.com/questions/414572/drawing-a-line-on-the-right-margin
\usepackage{eso-pic}
\usetikzlibrary{positioning}

%\AtBeginDocument{
%\AddToShipoutPictureBG{%
%  \AtPageLowerLeft{%
%    \begin{tikzpicture}[overlay,remember picture]
%    % angles
%    \coordinate (pagene) at (.98\paperwidth,.98\paperheight);
%    \draw[thin] (pagene) -- +(-1,0) -- (pagene) -- +(0,-1);
%    \coordinate (pagenw) at (.02\paperwidth,.98\paperheight);
%    \draw[thin] (pagenw) -- +(1,0) -- (pagenw) -- +(0,-1);
%    \coordinate (pagese) at (.98\paperwidth,.02\paperheight);
%    \draw[thin] (pagese) -- +(-1,0) -- (pagese) -- +(0,1);
%    \coordinate (pagesw) at (.02\paperwidth,.02\paperheight);
%    \draw[thin] (pagesw) -- +(1,0) -- (pagesw) -- +(0,1);
%    \end{tikzpicture}
%  }
%  \ifthenelse{\value{page}>0}{% if page > 2 add:
%  \AtPageLowerLeft{%
%    \begin{tikzpicture}[overlay,remember picture]
%    % Writing at the very top
%    \node[anchor=north east, align=center,
%    font=\small] (W)
%    at (.95\paperwidth,.95\paperheight)
%    {};
%    \node[anchor=north east,
%    font=\bfseries, align=right]
%    at (W.north west) {};
%    % vertical lines
%    %\draw[thin] (W.north west) -- +(0,-.9\paperheight);
%    %\draw[thin] (W.north east) -- +(0,-.9\paperheight);
%    % horizontal line
%    %\draw[thin] (W.south west) -- (W.south east);
%    \end{tikzpicture}}%
%  }{}%
%  }
%}

%%% page layout
%\usepackage{layouts}
%\AtEndDocument{
%\printinunitsof{mm}
%\twocolumnlayoutfalse%
%\currentpage%
%\marginparswitchtrue%
%\pagedesign%
%\thispagestyle{empty}
%\newpage%
%}

%%% cover page
\newcommand*{\coverpage}{
  \begingroup%
    \rule{\textwidth}{1pt}\par
    \vspace{0.4\baselineskip}
    {\sffamily\Huge\bfseries Final Report}\\
    \rule{\textwidth}{1pt}\par
    \vspace*{10ex}
    {\Large \xtitle}\\[0.5\baselineskip]
    {\Large NPRP \#\xnprp}\par
    \vspace*{7ex}
    \xinvestigators
    \vspace*{4ex}
    \xaffiliations
    \vfill%
    %{\xdate}
    \begin{tabular}{ll}
      Start date: & \xstartdate\\
      End date: & \xenddate%\\
      %Report date: & \xdate\\
    \end{tabular}
  \endgroup
  \clearpage%
  }

  %%% frontmatter
  \newcommand*{\makecover}{%
  \frontmatter%
  \addcontentsline{toc}{chapter}{Cover Page}
  \thispagestyle{empty}
  \coverpage%

  % \renewcommand\Authsep{\\}
  % \renewcommand\Authand{\\}
  % \renewcommand\Authands{\\}
  % \maketitle%
  % \thispagestyle{empty}

  \chapter*{Disclaimer}\label{chp:disclaimer}
  \addcontentsline{toc}{chapter}{Disclaimer}
  This project outcomes report for the general public is displayed verbatim as submitted by the Principal Investigator[s] (PI's) for this award. Any opinions, findings, and conclusions or recommendations expressed in this report are those of the PI’s and do not necessarily reflect the views of the Qatar National Research Fund (QNRF); QNRF has not approved or endorsed its content.
  %\clearpage%

  \vspace{10ex}
  {\let\clearpage\relax\chapter*{Acknowledgments}\label{chp:acknowledgments}}
  %\chapter*{Acknowledgments}\label{chp:acknowledgments}
  \addcontentsline{toc}{chapter}{Acknowledgments}
  This project was made possible by a National Priorities Research Program (NPRP) grant \#\xnprp{} from the Qatar National Research Fund, a member of Qatar Foundation. The statements made herein are solely the responsibility of the author[s].
  \clearpage%

  %\begin{KeepFromToc}
  \tableofcontents%
  \label{chp:contents}
  %\end{KeepFromToc}
  \clearpage%

  \listoftables%
  \label{chp:tables}
  \clearpage%

  \listoffigures%
  \label{chp:figures}
  \clearpage%

  \chapter*{List of Acronyms}\label{chp:acronyms}
  \addcontentsline{toc}{chapter}{List of Acronyms}
  \par%
  \printglossary[type=acronym,title=List of Acronyms,nonumberlist]

  \chapter*{Executive Summary}\label{chp:summary}
  \addcontentsline{toc}{chapter}{Executive Summary}
  \xsummary%

  \mainmatter%
}

\AtBeginDocument{
  \makecover%
}

%%% glossaries: acronyms (abbreviations) and symbols (nomenclatures)
%\input{\jobname.glossaries}
% Hamad Medical Corporation
% Georges Younes

\newglossary[nmg]{nomenclature}{nms}{nmo}{Nomenclatures}
\renewcommand{\glossarysection}[2][]{}

\makeglossaries%
% \makeindex%

\glsaddall[types={acronym,nomenclature}]
%%% http://theoval.cmp.uea.ac.uk/~nlct/latex/packages/glossaries/glossaries-manual.html#SECTION000312000000000000000
%%% https://www.dickimaw-books.com/gallery/glossaries-styles/
\glossarystyle{super4col}
\renewcommand*{\glsgroupskip}{}

%%% abbreviations

\newacronym{3d}{3D}{Three-Dimensional}
\newacronym{api}{API}{Application Programming Interface}
\newacronym{aua}{AUA}{American Urological Association}
\newacronym{aub}{AUB}{American University of Beirut}
\newacronym{bvh}{BVH}{Bounding Volume Hierarchy}
\newacronym{cad}{CAD}{Computer-Aided Design}
\newacronym{ccd}{CCD}{Continuous Collision Detection}
\newacronym{cpu}{CPU}{Central Processing Unit}
\newacronym{ct}{CT}{Computed Tomography}
\newacronym{dcd}{DCD}{Discrete Collision Detection}
\newacronym{dicom}{DICOM}{Digital Imaging and Communications in Medicine}
\newacronym{dp}{DP}{Double Precision}
\newacronym{fem}{FEM}{Finite Element Method}
\newacronym{gpgpu}{GPGPU}{General Purpose Graphics Processing Unit}
\newacronym{gpu}{GPU}{Graphics Processing Unit}
\newacronym{gui}{GUI}{Graphical User Interface}
\newacronym{hmc}{HMC}{Hamad Medical Corporation}
\newacronym{hmd}{HMD}{Head-Mounted Display}
\newacronym{hmv}{HMV}{Hierarchical Matrix-Vector Multiplication}
\newacronym{hud}{HUD}{Head-Up Display}
\newacronym{ip}{IP}{Intellectual Property}
\newacronym{isi}{ISI}{Intuitive Surgical Inc\onedot}
\newacronym{mis}{MIS}{Minimally-Invasive Surgery}
\newacronym{mmvr}{MMVR}{Medicine Meets Virtual Reality}
\newacronym{mri}{MRI}{Magnetic Resonance Imaging}
\newacronym{ovm}{OVM}{Open Volume Mesh}
\newacronym{pbdd}{PBDD}{Position-Based Deformable Dynamics}
\newacronym{pi}{PI}{Principal Investigator}
\newacronym{ppl}{PPL}{Pubo-Prostatic Ligament}
\newacronym{rarp}{RARP}{Robot-Assisted Radical Prostatectomy}
\newacronym{sah}{SAH}{Surface Area Heuristic}
\newacronym{sp}{SP}{Single Precision}
\newacronym{ui}{UI}{User Interface}
\newacronym{unc}{UNC}{University of North Carolina at Chapel Hill}
\newacronym{usa}{USA}{United States of America}
\newacronym{us}{US}{Ultrasound}
\newacronym{ut}{UT}{Urethral Transection}
\newacronym{vrh}{VRH}{Virtual Reality Headset}
\newacronym{vr}{VR}{Virtual Reality}
\newacronym{xml}{XML}{Extensible Markup Language}
\newacronym{ode}{ODE}{Ordinary Differential Equation}

%%% nomenclatures

%\newglossaryentry{U}{ % the label
%name={universal set}, % the term
%description={the set of all things}, % a brief description
%symbol={\ensuremath{\mathcal{U}}}  % the associated symbol
%}

%\newglossaryentry{matrix}{ % the label
%name=matrix, % the term
%description={a rectangular table of elements}, % brief description
%plural=matrices % the plural
%}

\newglossaryentry{c}{
type=nomenclature,
name={\ensuremath{c}},
%symbol={\ensuremath{c}},
sort={0},
description={speed of light}
}

\newglossaryentry{iteration}{
type=nomenclature,
name={\ensuremath{\cdot^{[k]}}},
%symbol={\ensuremath{\cdot^{[k]}}},
sort={1},
description={value of $\cdot$ at iteration $k$}
}

\newglossaryentry{einstein}{
type=nomenclature,
name={\ensuremath{a_ib_i}},
sort={2},
description={Einstein summation}
}

\newglossaryentry{pi}{
type=nomenclature,
name={\ensuremath{\pi}},
%symbol={\ensuremath{\pi}},
sort={3},
description={ratio of a circle's circumference to its diameter}
}

\newcommand{\acr}[1]{{\fontshape{ui}\selectfont\gls{#1}}}
\newcommand{\nom}[1]{\gls{#1}}

%%% phantom sub captions for referencing
% https://tex.stackexchange.com/questions/412838/memoir-label-and-ref-subfigures-inside-bigger-image

\makeatletter
% \begin{macro}{\subcaptionphantom}
% \cs{subcaptionphantom}\oarg{list-entry}\marg{caption} is a hidden
% non-printed subcaption. Designed for the case if "(a)", "(b)" are
% already embedded in the figure itself.
% Roughtly equivalent to \cs{phantomsubcaption} from the \Lpack{subcaption} package.
%
%    \begin{macrocode}
\newcommand{\subcaptionphantom}{%
  \bgroup
    \let\label=\memsub@label
    \ifdonemaincaption\else
      \advance\csname c@\@captype\endcsname\@ne
    \fi
    \refstepcounter{sub\@captype}\@contkeep
    \@ifnextchar [%
      {\@memsubcapphantom{sub\@captype}}%
      {\@memsubcapphantom{sub\@captype}[\@empty]}}
%    \end{macrocode}
% \end{macro}

% \begin{macro}{\@memsubcapphantom}
% Quick-and-dirty analog of \Lpack{memoir} \cs{memsubcap}, adapted
% for use in \cs{subcaptionphantom}.
%    \begin{macrocode}
\long\def\@memsubcapphantom#1[#2]#3{%
  \@tempdima=\hsize
  \vskip\subfloatcapskip
  \ifx \@empty #2
    \@memsubcaptionphantom{#1}{#3}{#3}%
  \else
    \@memsubcaptionphantom{#1}{#2}{#3}%
  \fi
  \vskip\subfloatcapskip
  \egroup}
%    \end{macrocode}
% \end{macro}
%
%
% \begin{macro}{\@memsubcaptionphantom}
% Quick-and-dirty analog of \Lpack{memoir} \cs{memsubcaption}, adapted
% for use in \cs{memsubcapphantom}.
%    \begin{macrocode}
\newcommand{\@memsubcaptionphantom}[3]{%

  \ifx \relax#2\relax \else
    \bgroup
      \let\label\@gobble
      \let\protect\string
      \def\@memsubcaplabel{\@nameuse{@@the#1}}%
      \xdef\@memsubfigcaptionlist{%
        \@memsubfigcaptionlist,%
  {\protect\numberline{\@memsubcaplabel}\noexpand{\ignorespaces #2}}}%
    \egroup
  \fi
  \@makesubfloatcaptionphantom{\@nameuse{@the#1}}{#3}%
  }

%    \end{macrocode}
% \end{macro}

% \begin{macro}{\@makesubfloatcaptionphantom}
% Quick-and-dirty analog of \Lpack{memoir} \cs{makesubfloatcaption}, adapted
% for use in \cs{memsubcaptionphantom}.
%    \begin{macrocode}
\newcommand{\@makesubfloatcaptionphantom}[2]{%
  \setbox\@tempboxa\hbox{%
    \@subcapsize
    {\phantom{\@subcaplabelfont#1}}{\ignorespaces #2}\unskip}%
  \@tempdimb=-\subfloatcapmargin
  \multiply\@tempdimb\tw@
  \advance\@tempdimb\@tempdima
  \hb@xt@\@tempdima{%
    \hss
    \ifdim \wd\@tempboxa >\@tempdimb
      \phantom{\memsubfig@caption{#1}}{#2}%
    \else
      \if@shortsubcap
        \phantom{\memsubfig@caption{#1}}{#2}%
      \else
        \box\@tempboxa
      \fi
    \fi
    \hss}}
%    \end{macrocode}
% \end{macro}
\makeatother



\DeclareCaptionFormat*{capstyle}{\noindent\hrulefill\par\bfseries#1.\space\normalfont#3\par}
\captionsetup[lstlisting]{format=capstyle,singlelinecheck=false}

%%% C code formatting lst
\lstdefinestyle{CStyle}{
  language=C,                % choose the language of the code
  numbers=left,                   % where to put the line-numbers
  stepnumber=1,                   % the step between two line-numbers.
  frame=tb,
  numbersep=5pt,                  % how far the line-numbers are from the code
  backgroundcolor=\color{white},  % choose the background color. You must add \usepackage{color}
  showspaces=false,               % show spaces adding particular underscores
  showstringspaces=false,         % underline spaces within strings
  showtabs=false,                 % show tabs within strings adding particular underscores
  tabsize=2,                      % sets default tabsize to 2 spaces
  captionpos=t,                   % sets the caption-position to top
  breaklines=true,                % sets automatic line breaking
  breakatwhitespace=true,         % sets if automatic breaks should only happen at whitespace
  title=\lstname,                 % show the filename of files included with \lstinputlisting;
  float
}
\renewcommand\lstlistingname{Code}
\lstset{
  basicstyle=\footnotesize,
  numbers=none,
  %numbers=left,
  numberstyle=\scriptsize,
  stepnumber=1,
  numbersep=8pt,
  tabsize=3,
%   frame=tb,
  breaklines=true,
  inputencoding=utf8,
  %columns=fullflexible
}
